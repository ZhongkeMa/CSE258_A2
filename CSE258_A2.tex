%%%%%%%%%%%%%%%%%%%%%%%%%%%%%%%%%%%%%%%%%%%%%%%%%%%%%%%%%%%%%%%%%%%%%%%%%%%%%%%%
%2345678901234567890123456789012345678901234567890123456789012345678901234567890
%        1         2         3         4         5         6         7         8

\documentclass[letterpaper, 10 pt, conference]{ieeeconf}  % Comment this line out
                                                          % if you need a4paper
%\documentclass[a4paper, 10pt, conference]{ieeeconf}      % Use this line for a4
                                                          % paper

%\IEEEoverridecommandlockouts                              % This command is only
                                                          % needed if you want to
                                                          % use the \thanks command
%\overrideIEEEmargins
% See the \addtolength command later in the file to balance the column lengths
% on the last page of the document



% The following packages can be found on http:\\www.ctan.org
%\usepackage{graphics} % for pdf, bitmapped graphics files
%\usepackage{epsfig} % for postscript graphics files
%\usepackage{mathptmx} % assumes new font selection scheme installed
%\usepackage{times} % assumes new font selection scheme installed
%\usepackage{amsmath} % assumes amsmath package installed
%\usepackage{amssymb}  % assumes amsmath package installed
\usepackage{graphicx}
\usepackage[utf8]{inputenc} % allow utf-8 input
\usepackage[T1]{fontenc}    % use 8-bit T1 fonts

\usepackage{tabu}
\usepackage{hyperref}       % hyperlinks
\usepackage{url}            % simple URL typesetting
\usepackage{booktabs}       % professional-quality tables
\usepackage{amsfonts}       % blackboard math symbols
\usepackage{nicefrac}       % compact symbols for 1/2, etc.
\usepackage{microtype}      % microtypography
\usepackage{amsmath}


\title{\LARGE \bf
Rating Prediction of Clothes Based On Review Text And Size Assessment
}

%\author{ \parbox{3 in}{\centering Huibert Kwakernaak*
%         \thanks{*Use the $\backslash$thanks command to put information here}\\
%         Faculty of Electrical Engineering, Mathematics and Computer Science\\
%         University of Twente\\
%         7500 AE Enschede, The Netherlands\\
%         {\tt\small h.kwakernaak@autsubmit.com}}
%         \hspace*{ 0.5 in}
%         \parbox{3 in}{ \centering Pradeep Misra**
%         \thanks{**The footnote marks may be inserted manually}\\
%        Department of Electrical Engineering \\
%         Wright State University\\
%         Dayton, OH 45435, USA\\
%         {\tt\small pmisra@cs.wright.edu}}
%}

\author{Zhongke Ma - A53267693% <-this % stops a space
	\\Zijun Zhao - A53318821% <-this % stops a space
	\\Member 3
	
}


\begin{document}



\maketitle
\thispagestyle{empty}
\pagestyle{empty}


%%%%%%%%%%%%%%%%%%%%%%%%%%%%%%%%%%%%%%%%%%%%%%%%%%%%%%%%%%%%%%%%%%%%%%%%%%%%%%%%
\begin{abstract}

This report illustrates a model used to predict final rating by predicting suitable size and profiles from people. This model is implemented by Python.

\end{abstract}


%%%%%%%%%%%%%%%%%%%%%%%%%%%%%%%%%%%%%%%%%%%%%%%%%%%%%%%%%%%%%%%%%%%%%%%%%%%%%%%%
\section{Dataset}
	
	In this project, we used Clothing Fit Data from ModCloth and RentTheRunway. Some basic statistics and properties are as followings:
	Here is one piece example:

	\lstset{language=C}
	\begin{lstlisting}
	{"fit": "fit",  // the customer experience
	 "age": "16", // the age of the customer
	 "size": 4,  // the size rent by the customer
	 "item_id": "1063761",  // a hashed identifier for this item 
	 "rating": "10", // the rating of the clothes
	 "rented for": "party", // the reason for renting the clothes
	 "review_text": "This hugged in all the right places! It was a perfect dress for my event and I received so many compliments on it. Not to mention customer service was great getting this to me in less than 24 hours!",  // text of the review
	 "review_summary": "It was a great time to celebrate the (almost) completion of my first year of law school.", // summary of the review
	 "category": "sheath", // the category of the Clothes
	 "height": "5' 4\"", // the height of the customer
	 "user_id": "360448", // the ID of the user, this is a hashed user identifier 
	 "review_date": "December 14, 2015" // the date of posting the review
	 }
	\end{lstlisting}
	
	In order to find the precise size for every customer and improve prediction of accuracy, only the record that customers give fit feedback are selected to train the model. Thus, there are totally 192462 samples in the dataset. 
	In order to make full use of the dataset, we will try two extract different features from the the raw data. And these features can be mainly divided into two parts. One is the assessment of whether the size of the cloths bought by the guests fit them or not and the other is to do some data mining on the review texts provided by the guests to analyze their sentiment to their products. Some detailed information about how we model the prediction task and the methods we use to implement the task will be provided in Section 2 and section 3. We will now describe some interesting findings we found from the raw data, which motivate the design of our model in the following sections.
	
	\subsection{Features of Size Determinants and Size} 
	1. In the size prediction, every features which might be associated with the size should be taken into consideration. However, there is no feature which are contained in every record. In another word, features like age, height, weight, bust size, cup size and so on should influence the size but those features mentioned above are not provided by every customer. That is one of the interesting parts in the dataset. To make sure precise predictions, the average of each feature is used to fill up when this data is missing in a record.
	2. In the size prediction, the range of size is extremely large, which is from 1 to 58. In a similar way, some size determinants, including age, height, weight and cup size, encounter the same situation. In other word, there are some extreme values in the dataset. But fortuneately, the amount of extreme values is too small compared to the size of dataset. This is the second interesting part in the dataset. Thus, the impact of extreme values can be ignored.
	3. Some features, such as shoes size, only appears in few customers. First of all, we want to consider these parameters into size determinants. Since these parameters appears too little and they are lack of being representative, they are erased from the list of size determinants.

	\subsection{Review Text}
	In supervised text classification, different stratagies are tried to improve the accuracy. One part of the improvement is achieved by using proper feature extraction methods and the other part of the improvement is achieved by grid searching to find the regularization parameter that leads to the highest accuracy on the development set.
	
	
\subsection{The Description of The Data And Data Clean-up}

There are 192462 samples with ratings provided in the dataset and each of them is made of a piece of review and a corresponding rating, labeling the review with a extent of how they like their clothes.
The original reviews are quite noisy. As some samples from the training set show:

\textbf{Sample 1.} "I LOVED the look, BUT: The only size offered was the two, which fit me everywhere but the waist. I pushed through and wore it anyway, and it made eating and drinking impossible and bruised my ribs. That said, the four would have been too large in the bust and shoulders. I ended up having my friend discreetly unzip me a bit and tuck in the sides, which gave me a bit of a plunging back, but allowed me to breathe!"

\textbf{Sample 2.} "PROS: 

Beautiful

Lace is gorgeous

Back cut out is fun/flirty without looking unflattering

Perfect for many occasions


CONS: 

WAYYY too small. I wore 3 sizes bigger than normal

The dress that fit my stomach was way too big in my shoulders, but I made it work"



There are abbreviations, punctuations, numbers and even some modal particles in the raw data. The words are also case-sensitive. All these factors have a negative impact on the system. Therefore, the data should be cleaned up before feature extraction. To be detail, I remove the digits and punctuations, and make every word lowercase.



\section{Predictive Task}
According to the profiles provided by each customer, firstly we predict the suitable size for every customer. Secondly, compare the prediction to the actual size, take the difference and review text into consideration, the final rating of every purchase will be predicted ny this model finally.
Thus, there are two tasks in this model. The first task is predicting whether the size is suitable for every customer according to two profiles provided , such as height, weight, age, bust size, cup size and so on. And the second task is considering the difference bewteen the prediction and the actual size customers buy,
 and build a model of text mining, which extracts review texts and analyze sentiment, calculate TFIDF and so on in order to predict final rating of every customer.


\section{Model}
\subsection{Feature Extraction}
To improve the accuracy of the supervised text classification, I tried many feature extraction methods. The detailed results of the CountVectorizer method and TfidfVectorizer are shown in Tab 1 and Tab 2. The corresponding analysis are as the following:

\begin{table}[htb]
	\caption{CountVectorize with different feature extraction methods}  
	\begin{center}  
		\begin{tabu} to 1\textwidth{X[3,c]|X[1,b]|X[2,l]|X[3,c]|X[2,m]|X[1,c]}  
			%0.8\textwidth   为设置表格宽度  
			%X[c]      表示这一列居中,所占比例为1,相当于X[1,c]  
			%X[3,c]   表示这一列居中,所占比例为3,这列的宽度是X[c]列的3倍  
			\hline  
			Regularization Coefficient / C  &Basic             &Without stop words      &Words(more than 2 characters)    &Accents stripped     &Trigram\\  
			\hline  
			C=1    &0.77729       &0.75764           &0.76855    &0.77729      &0.77947\\  
			Best C    &0.78165      &0.76200           &0.77292    &0.77947      &0.78820\\  
			Vocabulary Size    &9882      &9602           &9381   &9862      &167716\\  
			
			\hline  
		\end{tabu}  
	\end{center}  
\end{table} 

1) \textbf{Remove stop words: }Since a basic intuition is that the stop words like "the", "or", "a", etc. are meaningless when classifying a review into positive and negative, it makes much sense to remove these features. And this will decrease the vobabulary size to some extend. However, the result shows that I am too naive. When the stop words are removed, the accuracy decreases, contrary to my expectation. The reason might be stop words play an important role in expressing one's idea. That is to say, some meanful words might be considered as stop words and removed. For example, "down" is a stop word, but in some reviews it can be used as "the restaurant let me down" to express negative motion.

2) \textbf{Remove words with no more than 3 characters: }In order to move some modal particles, I only count the words with more than 2 characters as the feature. However, the result proves that the method has poor performance. This is because some modal particles are useful when classifying a review. Besides, the words removed may include some non-modal particles because all the words with no more than 2 characters are removed.

3) \textbf{Strip accents: }To deal with the languages besides English, I try to strip the accents. And this also lead to worse peroformance. This is because when the accents are stripped, some languages may be mixed and become unrecognizable, which will decrease the accuracy.

4) \textbf{Use Trigram: }Using trigram is to consider the sequence of three word as well as two wordas as features, which means $(w_{i-2},w_{i-1},w_i)$ and $(w_{i-1},w_i)$ are both considered as features. Therefore, as can be seen in Table 1, the feature size is much larger than other mothods. Since the words are more related with the words before and after, the features contain more information about the words' meaning. Therefore, it makes sense to achieve better performance.

5) \textbf{Use basic TfidfVectorizer: }TF-idf stands for Term frequency-inverse document frequency. Since the importance of a word increases proportionally to the number of times a word appears in the review but is offset by the frequency of the word in the corpus, the features' importance should be reweighted. The expression of a feature's IDF value is expressed as:
\begin{equation}
IDF(w)=log\frac{N}{N(w)}
\end{equation}
where N is the number of reviews in the corpus and $N(w)$ is the number of the reviews that contain the word $w$. When added smoothing method, the expression can be:
\begin{equation}
IDF(w)=log\frac{N+1}{N(w)+1}+1
\end{equation}
\begin{equation}
TFIDF(x)=TF(w)\times IDF(w)
\end{equation}
where $TF(w)$ is the frequency of word $w$ in a review. Since TFIDF method can decrease the importance of the meaningless words like "to, is, are", the feature will focus more on meaningful words like "nice, tasty, bad, etc." As we can see in Table 2, TFID can improve the accuracy greatly.

6) \textbf{TFIDF with sublinear TF: }The difference of sublinear TF with basic TF is that it transfers $TF(w)$ to $log[TF(w)]+1$. The result shows that sublinear TF has better performance in this project. 



\begin{table}  
	\caption{TfidfVectorizer with different feature extraction methods}  
	\begin{center}  
		\begin{tabu} to 1\textwidth{X[3,c]|X[1,b]|X[2,l]|X[3,c]|X[2,m]}  
			%0.8\textwidth   为设置表格宽度  
			%X[c]      表示这一列居中,所占比例为1,相当于X[1,c]  
			%X[3,c]   表示这一列居中,所占比例为3,这列的宽度是X[c]列的3倍  
			\hline  
			Regularization Coefficient / C  &Basic             &sublinear tf      &Words(more than 2 characters)         &Trigram\\  
			\hline  
			C=1    &0.76637       &0.76637           &0.76529    &0.77729      \\  
			Best C    &0.79475      &0.80131           &0.79039    &0.79476     \\  
			Vocabulary Size    &9882      &9882           &9381         &167716\\  
			
			\hline  
		\end{tabu}  
	\end{center}  
\end{table} 

\subsection{The Hyper-parameters of the Classifer}
\textbf{Description of Logistic Regression:}
The expression of logistic regression is actually a sigmoid function.
$$
g(x)=\frac{1}{1+e^{-wx}}
$$
The training of a logistic regression is to find the best weights $w$ that make the loss function minimized.
$$
L(w)=\Sigma_{i=1}^n-y_ilog[1+e^{-y^{(i)}(wx^{(i)})}]
$$
where n is the number of samples in the training set and $y_i$ is the label of $i_th$ data. When using a trained logistic regression classifer to classify, it computes the value of $wx$. When $wx>0$, $g(x)>0.5$, which means the data belongs to class 1 and when $wx<0$, $g(x)<0.5$, classifying into class 0. 

\textbf{Grid Search to Find The best Regularization Parameter: }

When using the logistic regression, some parameters are set in Table 3:

\begin{table}[h]
	\caption{Parameters of the logistic regression classifier}  
	\begin{tabular*}{12cm}{llllll}  
		\hline  
		Patameter & Penalty  & Dual Form &fit intercept  &class weight &solver\\  
		\hline  
		Value  & 'l2' & False &True &None &lbfgs\\    
		\hline  
	\end{tabular*}  
\end{table}   

To find the best value of the regularization parameter C, I grid seach from 0.1 to 50 with the step 0.1. The results of CountVectorizer and TFIDFVectorizer with the variation of C are shown in Fig 1 and Fig 2.

%\begin{figure}[htbp]
%	\centering
%	\begin{minipage}[t]{0.48\textwidth}
%		\centering
%		\includegraphics[width=6cm,height=5cm]{Acc_tfidf_sublinear.eps}
%		\caption{Accuracy of TfidfVectorizer}
%	\end{minipage}
%	\begin{minipage}[t]{0.48\textwidth}
%		\centering
%		\includegraphics[width=6cm,height=5cm]{Acc_count_trigram.eps}
%		\caption{Accuracy of CountVectorizer}
%	\end{minipage}
%\end{figure}



\textbf{Conclusion: } Considering all the feature extraction methods and hyper-parameter search, the best performance the system can achieve is implemented with the following: 

Using TfidfVectorizer with sublinear TF and set the regularization parameter $C=4.8$, the accuracy of the system can be 80.131.




\section{Literature}




\section{Results}

The resulting difference may not seem significant but it is a difference nonetheless. Furthermore, making the lexer with Lex proved to be less time consuming than programming it in C. With all this considered we can safely conclude that the lexer produced with Lex has a better performance than the one programmed in C, which may be because of the use of regular expressions and finite automata.


\addtolength{\textheight}{-12cm}   % This command serves to balance the column lengths
                                  % on the last page of the document manually. It shortens
                                  % the textheight of the last page by a suitable amount.
                                  % This command does not take effect until the next page
                                  % so it should come on the page before the last. Make
                                  % sure that you do not shorten the textheight too much.

%%%%%%%%%%%%%%%%%%%%%%%%%%%%%%%%%%%%%%%%%%%%%%%%%%%%%%%%%%%%%%%%%%%%%%%%%%%%%%%%



%%%%%%%%%%%%%%%%%%%%%%%%%%%%%%%%%%%%%%%%%%%%%%%%%%%%%%%%%%%%%%%%%%%%%%%%%%%%%%%%



%%%%%%%%%%%%%%%%%%%%%%%%%%%%%%%%%%%%%%%%%%%%%%%%%%%%%%%%%%%%%%%%%%%%%%%%%%%%%%%%


\begin{thebibliography}{99}

\bibitem{c1} Decomposing fit semantics for product size recommendation in metric spaces
Rishabh Misra, Mengting Wan, Julian McAuley
RecSys, 2018

\end{thebibliography}

\end{document}